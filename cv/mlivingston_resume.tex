% LaTeX file for resume 
% This file uses the resume document class (res.cls)

\documentclass{res} 
\usepackage{verbatim}
\usepackage[left=1.cm, right=1cm, top=1cm, bottom=1cm]{geometry}
\usepackage[T1]{fontenc}
\usepackage{lmodern}
\usepackage{comment}
\usepackage{hyperref}
\hypersetup{colorlinks,urlcolor=blue}
% \newcommand{\comment}[2]{#2}

% \usepackage{garamond}

% the margin option causes section titles to appear to the left of body text 
\resumewidth=7.0in % increase textwidth to get smaller right margin
% \usepackage{helvetica} % uses helvetica postscript font (download helvetica.sty)
% \usepackage{newcent}   % uses new century schoolbook postscript font 
\newsectionwidth{20pt} 

\begin{document} 
\setlength{\parskip}{10pt}
\renewcommand{\labelitemi}{\scriptsize$\bullet$} 

\name{{\huge Max Livingston} \hspace{73 mm} maxliving@gmail.com} 
\address{\hspace{2pt}} % needed to make url right-justified
\address{\href{http://maxlivingston.org}{maxlivingston.org}}

\begin{resume} 
  \section{Employment}
  {\bf Freebird} \\ 
  \null {\it Director of Data Science} \hspace{3pt} (Jan 2020 -- March 2020) \\
  \null {\it Senior Data Scientist} \hspace{17pt} (Feb 2019 -- Jan 2020) \\
  \null {\it Data Scientist} \hspace{49pt} (Feb 2017 -- Feb 2019)
  \begin{itemize} \itemsep 2.0pt \parskip 2.0pt %reduce space between items
    \item As the first Data Scientist, was solely responsible for framing Freebird's risk modeling approach, determining research directions, and producing actionable output for Finance and Business Development teams
    \item Created automated risk reporting pipeline in Python to collect data, ensemble model predictions, and produce outputs for B2B pricing and capital management
    \item Produced distributional forecasts of payouts to enable pricing of a \$5 million contract with a major credit card issuer
    \item Fit scalable Bayesian models using MCMC and Variational Inference to predict the probability of flight cancellations, delays, and last-minute prices
    \setlength{\parskip}{-2pt} 
    \begin{itemize}
      \setlength{\itemsep}{1.0pt}
    \item Presented research comparing methods for flight delay prediction at PAPIs conference 2018 (\href{https://www.infoq.com/presentations/flight-bayesian-prediction}{video})
    \end{itemize}
    \setlength{\parskip}{2.0pt} 
  \item Maintain Scala service to ingest real-time flight status updates from multiple third party sources so Freebird can notify travellers as soon as something happens to their flight
  \end{itemize}

  {\bf Knewton} \\ 
  \null {\it Data Scientist} \hspace{3pt} (May 2014 -- Sep 2016)
  \begin{itemize} \itemsep 2.0pt %reduce space between items
  \item Wrote production code in Java implementing online machine learning algorithms to deliver real-time education content recommendations
  \item Contribute to in-house Python library for researching student proficiency using Item Response Theory and Bayes Nets (portions of which are open-source here: \href{https://github.com/Knewton/edm2016}{https://github.com/Knewton/edm2016})
  % \item Specific projects include 
  %   \setlength{\parskip}{-1.75pt} 
  %   \begin{itemize}
  %     \setlength{\itemsep}{1.5pt}
  \item Built new model to predict how long students spend on pieces of content using a Gaussian Mixture Model
    %\item Modeling the strength of concept pre-requisite relationships to produce better remediation strategies and provide insights into publisher content
  \item Re-wrote a key component of the recommendation engine to simplify its logic and drastically improve its speed
    %\item Co-leading a weekly dashboard meeting of approximately 10 data scientists and engineers to better understand how our exisiting models are performing and track predictive accuracy metrics
    
  %  \end{itemize}
  \end{itemize}

  {\bf Federal Reserve Bank of New York}  \\
  \null {\it Senior Research Analyst} \hspace{3pt} (Jan 2013 -- May 2014)\\
        {\it Research Analyst} \hspace{20pt} (July 2012 -- Jan 2013)
  \begin{itemize} \itemsep 2.0pt %reduce space between items
  \item Performed econometric analyses in Stata, SAS, and R to aid economic research projects and inform monetary policy decisions.
  \item Published multiple research papers on the effect of the recession and stimulus on school district finances.
  \end{itemize}
  
  \section{Selected Publications} 
  \href{http://proceedings.mlr.press/v82/vandal18a.html}{``Prediction and Uncertainty Quantification of Daily Airport Flight Delays,''} with Thomas Vandal, Camen Piho, and Sam Zimmerman, {\it Proceedings of Machine Learning Research}, 82:45-51, 2018.

  \href{http://bit.ly/20jcEax}{``Did Cuts in State Aid During the Great Recession Lead to Changes in Local Property Taxes?,''} with Rajashri Chakrabarti and Joydeep Roy, {\it Education Finance and Policy}, 9(4), pp. 383--416. Fall 2014.


  %\href{http://bit.ly/1fFXJ3t}{``My Two (Per)cents: How Are American Workers Dealing with the Payroll Tax Hike?,''} with Basit Zafar and Wilbert van der Klaauw, {\it Liberty Street Economics Blog}, May 2013.

  A complete list is available at: \href{http://maxlivingston.org}{maxlivingston.org}

\section{Education} 
  {\bf Wesleyan University,} Middletown, CT \hfill September 2008 -- May 2012
  \begin{itemize} \itemsep -2pt
  \item Awarded High Honors in Economics for thesis analyzing the effect of teachers' unions on school district performance.
  \item BA in Economics; Phi Beta Kappa, Dean's List
  \end{itemize}

  {\bf Stuyvesant High School,} New York, NY \hfill September 2004 -- June 2008

  \section{Skills}
  \begin{itemize} \itemsep -2pt
  \item Python (including numpy, scipy, pandas), Scala, Java, SQL, Stata, GIS, \LaTeX, AWS, Git
  \end{itemize}

\end{resume} 

\end{document} 



